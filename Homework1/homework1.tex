\documentclass[letterpaper,11pt]{article}

\usepackage{geometry, pslatex, fancyhdr, graphicx}
\usepackage{amsmath,amsthm,amssymb,scrextend}
\usepackage{multicol}
\usepackage{tabularx}
\usepackage[makeroom]{cancel}
\usepackage{color}
\geometry{ margin = 1.0in }

%%% TODO modify these variables as per your homework %%%
\def\homeworknum{1}
\def\myname{Harshit Jain}
\def\myuserid{hmj5262}
%%%%

\pagestyle{fancy}
\lhead{{\bf CMPSC 464 Spring 2024}}
\chead{{\bf Assignment~\homeworknum}}
\rhead{{\bf \today}}
\let\newproof\proof
\renewenvironment{proof}{\begin{addmargin}[1em]{0em}\begin{newproof}}{\end{newproof}\end{addmargin}\qed}

\newcounter{problemid}
\stepcounter{problemid}
\def\newproblem{\clearpage\newpage{\bf Problem~\arabic{problemid}\stepcounter{problemid}}\hfill\par}

\setlength\parindent{0em} 
\setlength\parskip{8pt}
\setlength{\fboxsep}{6pt}


\begin{document}

\framebox[\textwidth]{
	\parbox{0.96\textwidth}{
		\parbox{0.12\textwidth}{\bf Name:}\parbox{0.6\textwidth}{\myname}\\
		\parbox{0.12\textwidth}{\bf User ID:}\parbox{0.6\textwidth}{\myuserid}
	}
}
%% your solutions %%%


% PROBLEM 1
\newproblem 
\begin{enumerate}
    \item 
    Let's suppose $\sqrt{13}$ is a rational number. Then we can write it $\sqrt{13} = \frac{a}{b}$ where $a, b$ are whole numbers, $b$ not zero.

    We additionally assume that this $\frac{a}{b}$ is simplified to lowest terms, since that can obviously be done with any fraction. Notice that in order for $\frac{a}{b}$ to be in simplest terms, both of $a$ and $b$ cannot be even. One or both must be odd. Otherwise, we could simplify $\frac{a}{b}$ further.

    From the equality $\sqrt{13} = \frac{a}{b}$ it follows that $13 = \frac{a^2}{b^2}$,  or  $a^2 = 13b^2$. Since $13$ is prime and $a^2$ is a multiple of $13$, then $a$ is multiple of $13$.

    If we substitute $a = 13k$ into the original equation $\sqrt{13} = \frac{a}{b}$, we get:

    $\Rightarrow (13k)^2 = 13b^2$

    $\Rightarrow b^2 = 13k^2$

    Since $13$ is prime and $b^2$ is a multiple of $13$ then $b$ is multiple of $13$.

    We now have a contradiction since $a$ and $b$ must have no common factors (except $1$) but we have proved that if $\frac{a}{b}$ exits then $a$ and $b$ must have common factor $13$.
    
    So $\frac{a}{b}$ can not exist and the square root of $13$ is irrational.
    
    \item Yes, we can prove that square root of any prime number is irrational.

    Let's suppose $\sqrt{p}$ is a rational number, where $p$ is any prime number. Let $\sqrt{p} = \frac{m}{n}$ where $m,n \in N$. and $m$ and $n$ have no factors in common.

    Now $p = \frac{m^2}{n^2}$,  or  $m^2 = pn^2$.
    
    Since $p$ is prime and $m^2$ is a multiple of $p$ then $m$ is multiple of $p$.
    
    If we substitute $m = pk$ into the original equation $\sqrt{p} = \frac{m}{n}$, we get:
    
    $\Rightarrow (pk)^2 = pn^2$

    $\Rightarrow n^2 = pk^2$
    
    Since $p$ is prime and $n^2$ is a multiple of $p$ then $n$ is multiple of $p$.
    
    We now have a contradiction since $m$ and $n$ must have no common factors (except $1$) but we have proved that if $\frac{m}{n}$ exits then $m$ and $n$ must have common factor $p$.
    
    So $\frac{m}{n}$ can not exist and the square root of any prime is irrational.

\end{enumerate}

% PROBLEM 2
\newproblem
$Q = \{ q_0, q_1, q_2, q_3 \}$

$q_0:$ Initial state, state after reading a digit that leaves a remainder of $0$ when divided by $4$ \\
$q_1:$ State after reading a digit that leaves a remainder of $1$ when divided by $4$ \\
$q_2:$ State after reading a digit that leaves a remainder of $2$ when divided by $4$ \\
$q_3:$ State after reading a digit that leaves a remainder of $3$ when divided by $4$

$\Sigma = \{ 0, 1, 2, 3 \}$

$q_0 = \{ q_0 \}$ (Start State)

$F = \{ q_0 \}$ (Accept State)

$\delta =$ 
\begin{tabular}{c|cccc}
    & 0 & 1 & 2 & 3 \\
    \hline
    $q_0$ & $q_0$ & $q_1$ & $q_2$ & $q_3$ \\
    $q_1$ & $q_0$ & $q_1$ & $q_2$ & $q_3$ \\
    $q_2$ & $q_0$ & $q_2$ & $q_0$ & $q_2$ \\
    $q_3$ & $q_0$ & $q_3$ & $q_2$ & $q_1$ \\
\end{tabular}

\includegraphics[scale = 0.20]{2}

% PROBLEM 3
\newproblem
\begin{enumerate}
    \item 


    \item 


\end{enumerate}

% PROBLEM 4
\newproblem


% PROBLEM 5
\newproblem
\begin{enumerate}
    \item 


    \item 


    \item 


\end{enumerate}
\end{document}